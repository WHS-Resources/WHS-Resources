\documentclass{article}
\usepackage[utf8]{inputenc}
\usepackage{amsmath}
\usepackage{geometry}
\geometry{
a4paper,
total={170mm,257mm},
left=20mm,
top=20mm,
}

\title{Thermodynamics}
\author{Neo Wang}
\date{\today}

\begin{document}

\maketitle

\section{Formulas}

\begin{itemize}
    \item Boyle's Law: $P_1V_1=P_2V_2$
    \item $k=1.38\times 10^{-23}J/K$
    \item Ideal Gas Law 1: $PV=nRT$
    \item Ideal Gas Law 2: $PV=NkT$ where $P$ is the absolute pressure of a gas, $V$ is the volume it occupies, $N$ is the number of atoms and molecules in the gas, and $T$ is its absolute temperature. $k$ is the \textit{Boltzmann constant}.
    \item Mechanical equivalent of heat: motion and heat are interchangable.
    \item Charles' Law: $$\frac{V_1}{T_1}=\frac{V_2}{T_2}$$
    \item First law of thermodynamics: $$\Delta U = Q - W$$
    \item $\propto$ means proportional to
    \item $T\propto KE$
    \item $$R=8.314\frac{J}{molK}$$
    \item $$P\Delta V=\textrm{Work}$$
    \item $$U = \frac{3}{2}nRT=\frac{3}{2}NK_BT$$
    \item $$e_{carnot} = 1 - \frac{T_C}{T_H}$$
\end{itemize}

\section{Solving}
\begin{itemize}
    \item Work done by an ideal gas (PV diagrams)
    \begin{itemize}
        \item $$W_{BY}=\int_{V_i}^{V_F}PdV$$
        \item Isochoric ($\Delta V = 0$)
        \item $W_{BY}=0J$. There is no area under the curve.
        \item Isobaric ($\Delta P = 0$)
        \item If P is constant then $$W_{BY}=P\Delta V=P(V_F-V_i)$$
        \item Isothermal ($\Delta T = 0$)
        \item Issue is temperature doesn't show up in the integral!
        \item Remember that if we have an ideal gas then $$W_{BY}=\int_{V_i}^{V_F}PdV=\int_{V_i}^{V_F}\frac{nRT}{V}dV=nRT\ln(\frac{V_F}{V_i})$$
        \item Adiabatic = (Q = 0)
        \item $$\Delta E_{int}=Q_{ON}-W_{BY}=W_{BY}=-\Delta E_{INT}=-nC_V\Delta T$$
        \item Note that this is steeper than the isotherm.
    \end{itemize}
    \item Carnot Cycle and Heat Engines
    \begin{itemize}
        \item $$e_{carnot} = 1 - \frac{T_C}{T_H}$$
        \item $$T_C=573K, T_H=773K$$
        \item If we plug this into the formula, we get $1-\frac{573}{773} = 26\%$
        \item A carnot engine is at temperatures 400K and 700K.
        \item If 14000J of heat is absorbed by the engine, how much heat is discarded into the cold reservoir. $Q_H=14000J$
        \item To solve this we use $$\frac{T_H}{T_C}=\frac{|Q_H|}{|Q_C|}$$
        \item Where T is temperature and Q is heat. H is our hot, and C is our cold.
        \item 
    \end{itemize}
\end{itemize}

\end{document}
